\documentclass{article}
\usepackage{amsmath}
\usepackage{bbm}
\usepackage[portuguese]{babel}
\usepackage{indentfirst}
\setlength{\parindent}{1.25cm}
\usepackage{xcolor}
\usepackage[utf8]{inputenc}
\title{Pandemia na Austrália - Modelagem Matemática dos dados epidemiológicos da COVID-19.}
\author{Gustavo Hermano Gusmão de Souza \\ gustavohermano3@gmail.com}
\date{\today}
%%%%%%%%%%%%%%%%%%%%%%%%%%%
\begin{document}
\maketitle
\section{Introdução}

\label{intro}

Austrália é um país que fica localizado na Oceania e possui cerca de 25.649.985  de habitantes. O país possui uma área com  7.741.220 Km² de extensão, tendo uma densidade populacional de aproximadamente 3,22 hab/km².

O país é divido em seis estados e dois território, sendo os três mais populosos New South Wales (NSW), Victoria e Queensland, com cerca de 8.157.700, 6.689.400 e 5.160.000 milhões de habitantes, respectivamente.   

Em meio a crise sanitária causada pela pandemia do novo \textit{coronavírus} no mundo, a Austrália se vê em uma situação atípica para o país que é a recessão econômica. A crise na Austrália, assim como no mundo todo, passa a não ser somente uma crise sanitária e começa a impactar outros setores, como turismo, energia, comércio internacional e o maior deles é o setor da economia. Ao logo de três décadas a Austrália vem crescendo economicamente e se vê obrigada a frear esse crescimento e começar a impedir que estragos maiores da crise sanitária no país prejudicassem a vida da população australiana. 

De acordo com Chang et al. (2020), a Austrália experimentou a maioria desses efeitos. O número de casos confirmados da COVID-19, ultrapassou 1.000 casos no dia 21 de março de 2020, e dobrando os números de casos em três dias. 

A partir disso, o governo australiano impôs uma série de medidas para evitar uma alta propagação do novo coronavírus. E este enfrentamento têm dado resultados positivos em relação a preservação de vidas e um número de infectados controlado.

Segundo Rockett et al. (2020) o governo australiano introduziu medidas progressivas de mitigação de epidemia a partir de 23 de março de 2020 para limitar as interações sociais, reduzir a propagação do vírus e prevenir as transmissões. Houve uma ampla realização de testes no estado de  NSW, com 1.541 testes realizados por cada 100.000 habitantes, sendo esta como uma das medidas anunciadas no estado.

Segundo informações descritas no site do governo do estado de NSW, foram realizados até o dia de 26 de setembro de 2020 um total de 2.672.392 testes com 4.029 casos de COVID-19 confirmados. Mas, o estado de Victoria que é o segundo estado mais populoso da Austrália, é onde se concentra a maior parte dos casos superando até então a marca de 20.000 casos. Neste, já foram realizados 2.661.989 milhões de testes.

Informações coletadas do próprio site do ministério da saúde do Governo australiano, descreve que sistema de saúde na Austrália é concebido por uma parceria público-privada por meio do programa \textit{Medicare}. O Medicare é o programa de seguro desaúde universal da Austrália. Assegura a toda população australiana, e também boa parte dos visitantes estrangeiros, acesso a uma ampla gama de serviços de saúde e hospitalares a um custo baixo ou gratuito. 

\section{Resumos}

\begin{center}
	\textbf{Artigo:} Modelling transmission and control of the COVID-19 pandemic in Australia
\end{center}

O artigo desenvolve baseando-se em um modelo com um agente de simulação computacional refinando casos da COVID-19 em progresso na Austrália. O modelo é regulado para reproduzir principais características da transmissão da Covid-19. 

Aplicação do modelo serve de comparação de várias medidas tomadas pelo Governo, Estado e população. Com isso, é possível mensurar os benefícios das medidas e o quão positivo elas se tornam. Por exemplo, restrições a viagens aéreas internacionais, isolamento de casos, fechamento de escolas, quarentena domiciliar, são medidas analisadas no modelo. 

\begin{center}
	\textbf{Artigo:} Stemming the flow: how much can the
	Australian smartphone app help to control
	COVID-19?
\end{center}   

O artigo descreve como um aplicativo lançado pelo Governo australiano pode interferir e diminuir a propagação do novo coronavírus. O \textit{\textbf{COVIDSafe app}} é um aplicativo que pode ser instalado em smartphone com sistemas Android e IOS, e ajuda a identificar pessoas que foram expostas ao coronavírus. 

O estudo sobre o aplicativo foi desenvolvido por meio de um modelo de sistemas dinâmicos. Eles levaram em consideração a forma que o coronavírus se propagava no país, processos de localização de casos e fatores que afetem a população a aceitação do aplicativo. 

O modelo de sistema dinâmico proposto 
teve uma estrutura dividido entre pessoas exposta-infectada-recuperada, utilizando dados da pandemia e informações divulgadas no país sobre o comportamento do vírus e, assim, determinar
valores de parâmetro. 

Para o modelo aplicado, foram examinados os fatores que influenciavam certas tendências projetadas, como o aumento do teste viral no país, a participação da comunidade nas redes sociais, 
distanciamento social entre a população, e o nível de adesão ao aplicativo COVIDSafe.

Dos resultados obtidos com a pesquisa foi que uma provável segunda onda da COVID-19 ocorrerá se o distanciamento social diminuir e a taxa de teste também diminuir. A intensidade dessa segunda onda também dependerá de como será a taxa de redução do distanciamento social e de testagem. 

Ao final, o trabalho conclui que a manutenção do distanciamento social, a ampliação em grande escala de teste para a COVID-19 são aliados com resultados favoráveis para a redução de casos. O COVIDSafe tem também potencial de ser um complemento importante para testes de COVID-19 e também como contribuir com o distanciamento social. Dependendo do nível de aceitação do aplicativo pela comunidade, ele poderia ter
um efeito mitigador significativo em uma segunda onda de COVID-19 na Austrália.

\section{Metodologia}
O número de casos acumulados de COVID-19 no início da pandemia teve um crescimento similar ao de uma função exponencial. Porém, com o tempo, o número de casos novos tendem a se estabilizar, e permanecer a um valor constante. Além disso, conforme a doença se espalha e medidas de isolamento social ou outras medidas que impeçam a evolução da doença começam a serem tomadas, haverá uma tendência de que os valores desacelerarem muito mais rápido e que voltem a acelerar se medidas de contenção forem flexibilizadas em um momento inoportuno.

Há um modelo matemático que descreve o crescimento populacional, chamado de \textit{\textbf{modelo logístico}}. Esse modelo poderá descrever melhor como o coronavírus se comporta no decorre do tempo. 

Definiremos P(t) o número de pessoas infectadas em função do tempo, tal que:

\begin{equation} \label{eq1}
	\left\{ \begin{split} \frac{dP}{dt} \approx KP\textrm{, se } P \textrm{ for pequeno, } \\
	\textrm{Se } P \rightarrow C, \textrm{ então } \frac{dP}{dt} \rightarrow 0.
\end{split} \right.
\end{equation}
sendo $K$ uma constante de proporcionalidade (positiva) e $C$ a capacidade de suporte. 

A equação (\ref{eq1}) acima, nos diz é que taxa de crescimento é proporcional a $ P $ quando se inicia o processo contaminação. E quando o número de infectados se aproxima de sua capacidade de suporte $ C $, a taxa de crescimento tende a $0$ e para de crescer, se estabilizando em $ C $.

A equação linear logística que conduz este modelo é dada por 
\begin{equation} \label{eq2}
	\frac{dP}{dt} = KP \cdot \left(  \dfrac{1-P}{C} \right).
\end{equation}

Trata-se de uma equação diferencial separável, então integrando ambos os lados da equação (\ref{eq2}), obtemos: 
\begin{equation}
	\int \dfrac{dP}{P\cdot \left(1-\frac{P}{C} \right)} = \int K\cdot dt.
\end{equation}
Daí, chegamos então na solução
\begin{equation}
	P(t) = \frac{C}{(1 + A \cdot e^{-Kt})}.
\end{equation}

\begin{thebibliography}{Bibliografia}
\bibitem{Chang} CHANG, Sheryl L. et al.,  Modelling transmission and control of the COVID-19 pandemic in Australia. 23 de mar. de 2020. Disponível em: https://arxiv.org/abs/2003.10218. Acesso em: 27, set. e 2020.
\bibitem{Danielle J Currie} CURRIE, Danielle J. et al.,  Stemming the fow: how much can the
Australian smartphone app help to control
COVID-19?. Jun. de 2020. Disponível em: https://www.phrp.com.au/wp-content/uploads/2020/06/PHRP3022009.pdf. Acesso em: 27, set. e 2020.
\bibitem{Rockett,}  ROCKETT, Rebecca J. et al.,  Revealing COVID-19 transmission in Australia by SARS-CoV-2 genome sequencing and agent-based modeling. 09 de jul. de 2020. Disponível em: https://www.nature.com/articles/s41591-020-1000-7. Acesso em: 27, set. e 2020.
\bibitem{Governo australiano} GOVERNO DA AUSTRÁLIA. National, state and territory population. Disponível em: https://www.abs.gov.au/statistics/people/population/national-state-and-territory-population/mar-2020 . Acesso em: 27, set. e 2020.
\bibitem{Governo de NSW} GOVERNO NEW SOUTH WALES. COVID-19 in NSW. Disponível em: https://www.health.nsw.gov.au/Infectious/covid-19/Pages/recent-case-updates.aspx. Acesso em: 27, set. e 2020.
\bibitem{Governo australiano} GOVERNO DA AUSTRÁLIA. Medicare. Disponível em: https://www.health.gov.au/health-topics/medicare. Acesso em: 27, set. e 2020.
\bibitem{Governo Victoria} GOVERNO VICTORIA. Coronavirus (COVID-19). Disponível em: https://www.dhhs.vic.gov.au/coronavirus. Acesso em: 27, set. e 2020.
\bibitem{Rockett,}  ROCKETT, Rebecca J. et al.,  Revealing COVID-19 transmission in Australia by SARS-CoV-2 genome sequencing and agent-based modeling. 09 de jul. de 2020. Disponível em: https://www.nature.com/articles/s41591-020-1000-7. Acesso em: 27, set. e 2020.
\end{thebibliography}
	
\end{document}